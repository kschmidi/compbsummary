\documentclass[11pt, oneside]{article}   	% use "amsart" instead of "article" for AMSLaTeX format
\usepackage{geometry}                		% See geometry.pdf to learn the layout options. There are lots.
\geometry{letterpaper}                   		% ... or a4paper or a5paper or ... 
%\geometry{landscape}                		% Activate for for rotated page geometry
%\usepackage[parfill]{parskip}    		% Activate to begin paragraphs with an empty line rather than an indent
\usepackage{graphicx}				% Use pdf, png, jpg, or eps§ with pdflatex; use eps in DVI mode
								% TeX will automatically convert eps --> pdf in pdflatex		
\usepackage{amssymb}

\usepackage[utf8]{inputenc}
\usepackage[nswissgerman]{babel}
\usepackage{hyperref}

\title{Compilerbau und Programmierkonzepte}
\author{Schmidiger Kevin}
\date{\today}						% Activate to display a given date or no date

\begin{document}
\maketitle
\newpage

%Inhaltsverzeichnis
\tableofcontents
\newpage

\section{Einf"uhrung}
Dieses Dokument ist weniger eine Zusammenfassung als eine Dokumentation von dem Modul Compilerbau und Programmierkonzepte. Ich dokumentiere und beschreibe gewisse Techniken und Vorgehensweisen.\\
In diesem Modul wird mit \hyperref[http://www.antlr.org/]{ANTLR} gearbeitet. Zu beginn werden kleine Parser geschrieben und dann erweitert.
%TODO improvements

\end{document}  